
\section{Introduction}
GMRES (Generalized Minimal Residual Method) and its variants based on iterative refinement (IR) are advanced numerical techniques for solving linear systems, particularly useful for large or ill-conditioned matrices where direct methods are inefficient or impractical. \\
GMRES is an iterative method for solving non-symmetric linear systems. It seeks to minimize the residual over a Krylov subspace generated by the matrix and the residual vector. The main steps in a GMRES iteration include the Arnoldi process to construct an orthogonal basis for the Krylov subspace, updating the QR factorization of the resulting Hessenberg matrix, and solving the resulting least squares problem to update the solution\cite{Homer1988}.\\
\begin{equation}
    Ax=b
\end{equation}
Assume that $\textbf{A}$ is a real nonsingular $\textbf{N}$ by$\textbf{N}$ matrix and $b$ is a real vector. Given an initial guess $x^0$ for the solution, the algorithm generates approximate solutions $x^n,n=1,2,...,\textbf{N}$ from the linear variety
\begin{equation}
    x^0+\textbf{K}_n(\textbf{A},r^o)
\end{equation}
minimizing the Euclidean norm of the residual,
\begin{equation}
    \| \mathbf{b} - A\mathbf{x}^n \| = \min_{\mathbf{u} \in \mathbf{x}_0 + \mathcal{K}_n(A,r_0)} \| \mathbf{b} - A\mathbf{u} \|,
\end{equation}

where $r^0=b-\textbf{A}x^0$ is the initial residual and $\textsc{K}_n(\textbf{A},r^0)$ is the $n-th$ Krylov subspace generated by \textbf{A},$r^0$,
\begin{equation}
    K_n(A, r^0) = \text{span}\{r^0, Ar^0, \dots, A^{n-1}r^0\}.
\end{equation}
Clearly, $r^n \in r^0 + A K_n(A, r^0)$. We call $\textbf{AK}_{n}(\textbf{A},r^0)$ a Krylov residual subspace. \\
Such an approximation always exists and is unique, It can be computed in many different ways. \\
Most methods for computing the approximation $x^n$ satisfying (3) start by constructing an orthonormal basis,called an Arnoldi basis, for the Krylove subspace  (4). The recurrence for the basis vectors can be written in matrix for as 
\begin{equation}
    AV_n=V_{n+1}H_{n+1,n}
\end{equation}
Where $V_{n+1}$ is the $N$ by $(n+1)$ matrix with the orthonormal basis vectors $v_1,V_2,\dots,V_{n+1}$ as its columns and $H_{n+1,n}$ is the $(n+1)$ by $n$ upper Hesenberg matrix of the orthogonalization (and normalization) coefficients, $n<N$. The initial vector $\mathbf{s} = \frac{\mathbf{r}^0}{\rho}, \text{ where } \rho = \|\mathbf{r}^0\|.$ The approximate solution $x^n$ is then taken to be of the form $x^n=x^0+V_ny^n$, where $y^n$ is cosen to minimize
\begin{equation}
    \| \mathbf{b} - A\mathbf{x}^n \| = \| \mathbf{r}^0 - AV_n\mathbf{y}^n \| \\
= \| V_{n+1}(\mathbf{e}_1 - H_{n+1,n}\mathbf{y}^n) \| \\
= \| (\mathbf{e}_1 - H_{n+1,n}\mathbf{y}^n) \|.
\end{equation}
The problem of solving apporximately the original $N$-dimensional system $Ax=b$ is thus transformed to the $n$-demensional least squares problem
\begin{equation}
    \| \mathbf{e}_1 - H_{n+1,n}\mathbf{y}^n \| = \min_{\mathbf{y}} \| \mathbf{e}_1 - H_{n+1,n}\mathbf{y} \|, \quad \mathbf{x}^n = \mathbf{x}^0 + V_n\mathbf{y}^n.
\end{equation}
We call $b-Ax^n$ a true residual, $\rho e_1 - H_{n+1,n}y^n$ an Arnoldi residual. While the norms of these two vector are the same in exact arithmetic.\\
There are sereval algorihms for computing the Arnoldi basis, whilce GMRES has played a significant role since it has been invented.\\
GMRES-IR extends GMRES by incorporating mixed precision strategies to improve computational efficiency and robustness. GMRES-IR operates by using different precision levels throughout the computation process. It typically involves a sequence where a lower precision is used for certain operations to save computational cost, while higher precision is reserved for critical steps to ensure the accuracy of the solution. This method is particularly effective when solving nearly singular or ill-conditioned systems where high precision is required to achieve an accurate solution .\\
\subsection{Key Concepts of GMRES and IR algorithms based on GMRES}
\begin{itemize}
    \item Multiple Precision Levels:
    \begin{itemize}
        \item GMRES-IR strategically uses different precision levels through the computation process. Lower precision is used where high accuracy is less crucial, reducing computational cost and memory usage.
        \item Higher precision is used in critical steps, particularly in the evaluation of residuals and the final solution steps, where maintaining numerical accuracy is paramount. This approach helps in managing the trade-off between computational speed and accuracy effectively.
    \end{itemize}
    \item Iterative Refinement: 
    \begin{itemize}
        \item In GMRES-IR, the initial solution to the linear system is typically computed using a standard GMRES iteration in a lower precision format (e.g., single precision). This serves to obtain a coarse solution quickly.
        \item Once this solution is computed, the residual (the difference between the left-hand side and right-hand side of the linear equation using the current solution) is recalculated in higher precision (e.g., double precision). This step is crucial because it allows the detection of small errors that lower precision may miss.
        \item The refined solution is then improved iteratively by applying additional GMRES cycles targeted at the residual, computed in higher precision. This process can be repeated multiple times to progressively enhance the solution accuracy.
    \end{itemize}
    \item Preconditioning: 
    \begin{itemize}
        \item Preconditioning in GMRES-IR is vital for improving the convergence rates of the iterative process. A preconditioner transforms the original system into one that is easier (i.e., faster) to solve iteratively.
        \item The preconditioner itself can be computed in a lower precision to save computational resources, and then applied during the higher precision solution phase. This is effective in reducing overall computational times while still maintaining a robust approach towards convergence.
    \end{itemize}
    \item Application-Specific Adjustments:
    \begin{itemize}
        \item The configuration of precision levels, the choice of preconditioner, and the number of refinement iterations can be tailored based on specific application needs and hardware capabilities. This customization allows GMRES-IR to be optimized for specific scenarios, such as sparse systems, systems arising in fluid dynamics, or systems that require high levels of precision due to their sensitivity.
    \end{itemize}
    \clearpage
\end{itemize}
